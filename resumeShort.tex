% resume.tex
%
% (c) 2002 Matthew Boedicker <mboedick@mboedick.org> (original author) http://mboedick.org
% (c) 2003-2007 David J. Grant <davidgrant-at-gmail.com> http://www.davidgrant.ca
% (c) 2007-2014 Todd C. Miller <Todd.Miller@courtesan.com> http://www.courtesan.com/todd
%
% This work is licensed under the Creative Commons Attribution-ShareAlike 3.0 Unported License. To view a copy of this license, visit http://creativecommons.org/licenses/by-sa/3.0/ or send a letter to Creative Commons, 171 Second Street, Suite 300, San Francisco, California, 94105, USA.

\documentclass[letterpaper,10pt]{article}

%-----------------------------------------------------------
\usepackage[empty]{fullpage}
\usepackage{color}
\definecolor{mygrey}{gray}{0.80}
\raggedbottom
\raggedright
\setlength{\tabcolsep}{0in}

% Adjust margins to 0.5in on all sides
\addtolength{\oddsidemargin}{-0.5in}
\addtolength{\evensidemargin}{-0.5in}
\addtolength{\textwidth}{1.0in}
\addtolength{\topmargin}{-0.5in}
\addtolength{\textheight}{1.0in}

%-----------------------------------------------------------
%Custom commands
\newcommand{\resitem}[1]{\item #1 \vspace{-3.5pt}}
\newcommand{\resheading}[1]{{\large \colorbox{mygrey}{\begin{minipage}{\textwidth}{\textbf{#1 \vphantom{p\^{E}}}}\end{minipage}}}}
\newcommand{\ressubheading}[4]{
\begin{tabular*}{7.0in}{l@{\extracolsep{\fill}}r}
		\textbf{#1} & #2 \\
		\textit{#3} & \textit{#4} \\
\end{tabular*}\vspace{-6pt}}
%-----------------------------------------------------------


\begin{document}
% \begin{table}[]
%     \centering
    
%     \\
% \end{table}
% \centering
\begin{tabular*}{7.5in}{l@{\extracolsep{\fill}}r}
    \textbf{\large Alexander Y. Liu}  & 703.220.5928 (cell)\\
    331281 Georgia Tech Station &  aliu338@gatech.edu \\
    Atlanta, GA 30332-1400 & Website: 2019aliu.github.io \\
    % \textbf{\large Alexander Y. Liu}  & ‪(470) 588-0631 (cell)\\
    % 9908 Mill Run Drive &  2019alexliu@gmail.com \\
    % Great Falls, VA 22066 & https://github.com/2019aliu \\
    % , https://user.tjhsst.edu/2019aliu/ \\
\end{tabular*}

\vspace{0.1in}

% \begingroup
%     \fontsize{12pt}{12pt}\selectfont
%     \textbf{Objective}: Seeking a fascinating internship in collaborative environment to solve challenging software development, cybersecurity, or AI/ML problems
% \endgroup
% \vspace{1em}

\resheading{Education}
\begin{itemize}
\item[]  % empty optional argument means no bullet point
    % \ressubheading{Georgia Institute of Technology}{GPA: 3.6 \hspace{1em} Atlanta, GA}{B.S. Computer Science, 2022}{August 2019 - Present}
    \ressubheading{Georgia Institute of Technology}{GPA: 3.6 \hspace{1em} Atlanta, GA}{B.S. Computer Science, 2021}{August 2019 - Present}
    % \ressubheading{Georgia Institute of Technology}{Atlanta, GA}{B.S. Computer Science, 2022 (Minor Biomedical Engineering)}{August 2019 - Present}
	\begin{itemize}
	    \resitem{Selected courses: Data Structures and Algorithms, Objects and Design, Honors Linear Algebra with Abstract Vector Spaces, Discrete Mathematics, Combinatorics, Statistics and Applications, Macroeconomics}
% 		\resitem{Selected courses: Microeconomics, Discrete Mathematics, Objects and Design, Data Structures and Algorithms, Honors Linear Algebra with Abstract Vector Spaces, Statistics and Applications}
	\end{itemize}
	\ressubheading{Thomas Jefferson High School for Science and Technology}{Alexandria, VA}{Main Interests: Computer Science, Neuroscience}{September 2016 - June 2019}
	\begin{itemize}
% 		\resitem{Senior Research Project: Exploration of Two-Dimensional Materials for Inhibition of the Calcitonin Gene-Related Peptide Pathway for Migraines}
        \resitem{Selected coursework: Artificial Intelligence, Mobile Application Development, Web Application Development, AP Computer Science A and Data Structures, Neuroscience Research Lab, Neurobiology, Research Statistics}
        % \resitem{Honors received: Siemens Regional Semifinalist, Virginia Science and Engineering Fair: 2nd in Bioengineering Category, AP Scholar with Distinction, Athletic Honor Roll}
% 		\resitem{Computer Science Courses Taken: Mobile Application Development, Web Application Development, Artificial Intelligence, AP Computer Science A Plus Data Structures, Accelerated Computer Science}
% 		\resitem{Neuroscience-Related Courses Taken: Neuroscience Research Lab, Neurobiology, AP Biology, Honors Biology}
	\end{itemize}
\end{itemize}

\resheading{Skills}

\begin{description}
\item[Languages:] Java, Python, JavaScript, Golang, SQL, HTML, CSS, LaTeX, Dart % Markdown, Bash are there, but not necessary
\item[Infrastructures and Frameworks:] Node.js, React.js, Bootstrap, Git, MongoDB/MongoDB Atlas, Firebase, Redis, Vue.js, Keras, Tensorflow, WebSocket, Heroku, MySQL, PyTorch, Angular, Flutter
\item[Software:] Terminal (Linux, Mac), Postman, Vim, Jupyter Notebook, Android Studio, Figma, Visual Studio Code, IntelliJ IDEA, PyCharm, jGRASP, Google Colabratory, Windows Subsystem for Linux (WSL)
% \item[Miscellaneous:]
% Great troubleshooting and debugging skills, great at explaining concepts to other people
\end{description}

\resheading{Selected Experience and Projects}

\vspace{1em}
The full list of projects I have worked on can be found on my GitHub: github.com/2019aliu

\begin{itemize}
\item
    \ressubheading{Software Developer Intern}{Rockville, MD - remote}{S\&C Electric}{May 2020 - Present}
    \begin{itemize}
        \resitem{Build a web application to view and edit settings of electrical products for consumers nationwide}
        \resitem{Design a proxy-microservice type application to allow for easier maintainence and construction}
        \resitem{Construct UI with React.js and supporting proxy with GraphQL and Apollo for client-side operations}
        \resitem{Implement, test, and document microservice to retrieve data from S\&C Electric's devices, and a microservice to open channels for subscribing to the devices' data}
        \resitem{Technologies used: Java, Javascript, GraphQL, Redis, Spring Boot, React.js, WebSocket and STOMP, Docker}
    \end{itemize}
% \item
%     \ressubheading{Software Developer Intern}{Greenbelt, MD}{Fluency Security Corporation}{June 2019 - August 2019}
% 	\begin{itemize}
% 	    \resitem{Developed a web-based trouble ticketing system, FasterIncidentResponse, using MongoDB-Gin-Vue.js-Golang fullstack framework, and integrated it into existing log management software}
% 		\resitem{Created developer's guide documentation with Postman, Markdown, and Web Developer tools}
% 		\resitem{Unit tested log management software with Golang's unit testing framework}
% 		\resitem{Technologies used: Golang (including Gin server), MongoDB, Bootstrap, Vue.js, Node-RED, Visual Studio Code}
% 	\end{itemize}

\item
    \ressubheading{TAG}{Atlanta, GA}{Create-X: Idea to Prototype}{January 2020 - Present}
    \begin{itemize}
        \resitem{Create a tracking device that has better range than most commercially available tracking tags by utilizing GPS/Bluetooth/Wifi technology}
        % \item Uses Global Positioning System (GPS) to determine vicinity of device, Bluetooth/Wifi to identity exact location, and Android Studio to create a mobile app to easily manage tracking
        \resitem{Technologies used: Android Studio, Java, XML, Google Nearby Messages API, Google Maps API}
    \end{itemize}
% \item
%     \ressubheading{CoronaDigest}{Charlottesville, VA}{HookHacks 2020}{March 28-29, 2020}
%     \begin{itemize}
%         \resitem{Make a web application that provides the latest news about Coronavirus (COVID-19), including a 2-minute daily digest, a 3D globe of Coronavirus cases, and financial information related to the Coronavirus.}
%         \resitem{Technologies used: Python, Plotly, Seaborn, Matplotlib, MongoDB Atlas, Pandas, Jupter Notebook, Flask, Jinja, Bootstrap}
%     \end{itemize}
% \item
%     \ressubheading{Creating the Next}{Atlanta, GA}{Hacklytics 2020}{February 22-23, 2020}
%     \begin{itemize}
%         \resitem{Visualized unemployment data and other macroeconomic factors nationally and globally, and built multivariate regression model to determine how much the government should spend on unemployment}
%         \resitem{Won best use of visualizations}
%         \resitem{Technologies used: Python, Plotly, Seaborn, Matplotlib, MongoDB Atlas, Pandas, Jupyter Notebook, Flask}
%     \end{itemize}
% \item
%     \ressubheading{FoxStocks}{Athens, GA}{UGAHacks 5}{February 7-9, 2020}
%     \begin{itemize}
%         \resitem{Created web application to teach new investors how to invest in stocks. My part was mostly backend work.} 
%         \resitem{Won best use of MongoDB Atlas}
%         \resitem{Technologies used: Flask, Jinja, MongoDB Atlas, Python, BlackRock Aladdin API}
%     \end{itemize}
% \item
%     \ressubheading{Tracker-X}{Atlanta, GA}{Create-X: Idea to Prototype}{January 2020 - Present}
%     \begin{itemize}
%         \item Create a tracking device that has better range than most commercially available tracking tags by utilizing GPS technology
%         % \item Uses Global Positioning System (GPS) to determine vicinity of device, Bluetooth/Wifi to identity exact location, and Android Studio to create a mobile app to easily manage tracking
%         \item Technologies used: Android Studio, Java, GPS, Google Maps API
%     \end{itemize}
% \item
%     \ressubheading{FoxStocks}{Athens, GA}{UGAHacks 5}{February 7-9, 2020}
%     \begin{itemize}
%         \item Created web application to teach new investors how to invest in stocks. My part was mostly backend work.
%         \item Won best use of MongoDB Atlas
%         \item Technologies used: Flask, Jinja, MongoDB Atlas, Python, BlackRock Aladdin API
%     \end{itemize}
    
    
% \item
%     \ressubheading{TimePlotter}{Atlanta, GA}{Big Data Big Impact Club}{October 2019 - Present}
%     \begin{itemize}
%         \item Develop a data analytic algorithm for a time-based plot of Atlanta using SGD technique to optimize the lengths between any two points based on the time taken
%         \item Technologies used: Pandas, Google Maps API, Python
%     \end{itemize}

% \item
%     \ressubheading{Season2Season}{Atlanta, GA}{Agency Club}{October 2019 - Present}
%     \begin{itemize}
%         \item Create a tool to change the season of an outdoors picture using a Generative Adversarial Network (GAN) machine-learning model trained with 1000+ images
%         \item Technologies used: PyTorch, Python
%     \end{itemize}

% \item
%     \ressubheading{Inline}{Durham, NC}{HackDuke 2020}{November 2-3, 2019}
%     \begin{itemize}
%         \item Created web application to search for nearby health centers with the specified treatments and sort them by transportation time using Google Maps API
%         \item Technologies used: Flask, Google Maps API, MongoDB Atlas, HTML/CSS/JS
%     \end{itemize}

% \item
%     \ressubheading{Stockastic}{Atlanta, GA}{HackGT 6}{October 25-27, 2019}
%     \begin{itemize}
%         \item Designed and implemented a web application that helps users to monitor stocks of their interest by conducting sentiment analysis of Twitter tweets about the corresponding companies
%         \item Technologies used: React.js, Express.js, HTML/CSS, MongoDB,  Twitter API, Google Cloud Natural Language API
%     \end{itemize}
\item
    \ressubheading{Software Developer Intern}{Greenbelt, MD}{Fluency Security Corporation}{June 2019 - August 2019}
	\begin{itemize}
	    \resitem{Developed a web-based trouble ticketing system, FasterIncidentResponse, using MongoDB-Gin-Vue.js-Golang fullstack framework, and integrated it into existing log management software}
		\resitem{Created developer's guide documentation with Postman, Markdown, and Web Developer tools}
		\resitem{Unit tested log management software with Golang's unit testing framework}
		\resitem{Technologies used: Golang (including Gin server), MongoDB, Bootstrap, Vue.js, Node-RED, Visual Studio Code}
	\end{itemize}
% \item
%     \ressubheading{Tetris: Forty Lines}{Alexandria, VA}{Mobile Applications Development}{March 2019 - June 2019}
% 	\begin{itemize}
% 		\resitem{Implemented a swipe-capable Tetris Android app in Android Studio with Java backend}
% 		\resitem{Technologies used: Android Studio, Java}
% 	\end{itemize}
	
% \item
%     \ressubheading{Swipe-based Tetris}{Alexandria, VA}{Mobile Applications Development}{March 2019 - June 2019}
% 	\begin{itemize}
% 		\resitem{Independent developer of swipe-based, instead of button-based, Android application for Tetris}
% 		\resitem{Majority of current mobile applications for Tetris are button-based, and is very awkward given the area of a smartphone}
% 		\resitem{Technologies used: Android Studio, Java}
% 	\end{itemize}
% \item
%     \ressubheading{LegiChat}{Alexandria, VA}{HackTJ 6.0}{April 2019}
% 	\begin{itemize}
% 		\resitem{Developer of the LegiChat hack for HackTJ 6.0}
% 		\resitem{Motivated by the lack of a unified method of contacting local Congresspeople, as well as the Phone2Action challenge.}
% 		\resitem{Technologies used: Phone2Action API, HTML, CSS, JS, Node.js, Python (for scrapping data, elastic search), Git}

% \item
%     \ressubheading{Arcade Game Suite}{Alexandria, VA}{Web Applications Development}{September 2018 - January 2019}
% 	\begin{itemize}
% % 		\resitem{Personal website with projects and coding exercises}
% 		\resitem{Designed and developed web-based suite of games, including U.S. Minesweeper, Tetris, and a word-finder assistant for Scrabble}
% % 		\resitem{Explored functionalities of Javascript, including making a standalone server, asynchronous programming, and integrating databases}
% % 		\resitem{Initially created for Web Applications Development course, later expanded website for other projects.}
% 		\resitem{Technologies used: HTML/CSS/JS (including jQuery, AJAX), SQL, Node.js}
% 	\end{itemize}
% % 	\end{itemize}

% \item
%     \ressubheading{Personal Website}{Alexandria, VA}{Web Applications Development}{September 2018 - January 2019}
% 	\begin{itemize}
% 		\resitem{Personal website with projects and coding exercises}
% 		\resitem{Arcade-style website featuring games, such as U.S. Minesweeper and Tetris, as well as game-assisting tools, such as a Scrabble word finder}
% 		\resitem{Explored functionalities of Javascript, including making a standalone server, asynchronous programming, and integrating databases}
% 		\resitem{Initially created for Web Applications Development course, later expanded website for other projects.}
% 		\resitem{Technologies used: HTML/CSS/JS (including jQuery, AJAX), SQL, Node.js}
% 	\end{itemize}
% \pagebreak

% \item
%     \ressubheading{Website Developer and Administrator}{Chantilly, VA}{Hope Chinese School}{August 2018 - December 2018}
% 	\begin{itemize}
% 	    \resitem{Helped develop and administer a new website for cultural and enrichment center serving 5000 users}
% 		\resitem{Former administrator of the website, managing a system of tens of thousands of users.}
% 		\resitem{Website: https://www.hopechineseschool.org}
% 		\resitem{Technologies used: HTML/CSS/JS, Django, SASS}
% 	\end{itemize}
% \pagebreak

% should I talk about Boggle here?
%\pagebreak
% \item
%     \ressubheading{Othello AI}{Alexandria, VA}{Artificial Intelligence}{December 2017 - January 2018}
%     \begin{itemize}
%         \resitem{Coded an AI that can intelligently play the classic board game Othello}
%         \resitem{Competed in an Othello AI competition}
%         % \resitem{Explored algorithms in AI, including BFS/DFS, minimax (and negamax), $\alpha$-$\beta$ pruning}
%         \resitem{Technologies used: Python}
%     \end{itemize}
% \item
%     \ressubheading{CardBot}{Alexandria, VA}{HackTJ 4.0}{March 2017}
%     \begin{itemize}
%         \resitem{Developed a proof-of-concept hack for finding best credit card options given user input from a Facebook Messenger chat-bot, used Capital One's API}
%         \resitem{Won Best Entrepreneurial Hack}
%         \resitem{Technologies used: Python, Facebook Messenger API, Capital One Hack-a-thon API}
%     \end{itemize}
\end{itemize}

% \pagebreak

\resheading{Research Experience}
\begin{itemize}
\item
	\ressubheading{Migraine Research}{Great Falls, VA}{Neuroscience Research Lab}{June 2018 - January 2019}
	\begin{itemize}
		\resitem{\textbf{Title}: Exploration of Two-Dimensional Materials for Inhibition of the Calcitonin Gene-Related Peptide Pathway in Migraines}
	    \resitem{Employed high-performance CPU cluster and slurm management in collaboration with high school's computer systems lab}
	    \resitem{Continued using ABINIT, an open-source package for making predictions about molecular systems based on solving quantum physics equations.}
	   % \resitem{Continue using the quantum physics-based, open-source package ABINIT}
	    \resitem{Research proposal accepted by neuroscience research lab at high school, received guidance and \$2400 funding for project}
	   % \resitem{Submitted to Intel Science Talent Search, presented at the Thomas Jefferson Symposium to Advance Research}
% 		\resitem{\textbf{Research Abstract}: \footnotesize{Current research shows blocking the Calcitonin Gene-Related Peptide Receptor (CGRPR) most effectively treats migraines. First-principle calculations have been performed to analyze the interaction between one of the most effective migraines medicines and the active amino acids in the CGRPR. Based on the premise that two-dimensional (2D) materials have van der Waals interactions with amino acids, computations on the binding energies between the active amino acids in CGRPR and the selected 2D materials, silicene, germanene, and graphene oxide, have been performed. Based on the calculated binding energies, the interaction strength of each selected 2D material with CGRPR and that of olcegepant with CGRPR were compared. Results indicate that silicene possesses potentially potency to treat migraines more effectively yet economically than most existing treatments do.}}
	\end{itemize}
\item
	\ressubheading{Alzheimer's Disease Research}{Alexandria, VA}{Project Lead}{June 2017 - August 2017}
	\begin{itemize}
	    \resitem{\textbf{Title}: Exploration of Chelation Materials for Treatment of Alzheimer's Disease}
% 		\resitem{Team lead for research regarding Alzheimer's Disease.}
		\resitem{Used ABINIT, an open-source package for making predictions about molecular systems based on solving quantum physics equations.}
		\resitem{Submitted to Siemens Competition 2017, achieved the semifinalist award}
% 		\resitem{Competed in science and engineering fairs, placed 2nd at the Virginia State Science and Engineering Fair}
% % 		\resitem{Responsibilities include: overseeing project experiment procedures.}
% % 		\resitem{\textbf{Research Abstract}: \scriptsize{First-principle calculations have been performed to investigate the interaction of different metal ions with amyloid-beta, along with adsorption of the metal ion by potential chelating materials. Binding energies were evaluated for metal interaction with a first coordination sphere consisting of three nitrogens and one oxygen. Results indicate that this coordination sequence possesses greatest compatibility for copper. Due to copper's strongest affinity, binding energies were also evaluated for its interaction with MoS2, WS2, reduced graphene oxide (rGO), and cyanide. Our results indicate cyanide and rGO to possess strong chelation potential for treatment of Alzheimer's disease.}}
	\end{itemize}
	
\end{itemize}

% \resheading{Community Leadership}
% % \begin{itemize}
% % \item
% % 	\ressubheading{Instructor of \textit{Introductory Computer Science}}{Chantilly, VA}{Hope Chinese School}{January 2015 - June 2019}
% % 	\begin{itemize}
% % 	    \resitem{Co-founded and instructed first computer science course in Hope Chinese School}
% % % 		\resitem{First computer science course created in HCS, inspired the creation of computer science classes at all locations in HCS}
% % 	   % \resitem{Learned leadership skills and social skills while reinforcing my computer science skills}
% % 	    \resitem{Outstanding service recognition for multiple years (2017, 2018) for voluntary service, received paid position in 2018-2019 school year}
% % 	\end{itemize}
% % \item
% % 	\ressubheading{NeuroInspire Inc.}{Alexandria, VA}{Instructor}{September 2016 - May 2017}
% % 	\begin{itemize}
% % 		\resitem{Instructor for the NeuroInspire outreach program and 2017 NeuroInspire Impulse event}
% % 		\resitem{Taught underpriveliged middle schoolers in the outreach program}
% % 		\resitem{Worked with TJ Partnership to acquire funding for thousands of dollars of equipment}
% % 	\end{itemize}
% % \item
% % 	\ressubheading{TJHSST Techstravaganza}{Alexandria, VA}{Volunteer, Instructor}{May 2018 and May 2019}
% % 	\begin{itemize}
% % 		\resitem{Taught young children neurosciences by performing brain dissections and having the young children ride a "reverse bike" (2018)}
% % 		\resitem{Continued this instruction by having the children make model neurons (2019)}
% % 	\end{itemize}
	
% \end{itemize}

\end{document}
